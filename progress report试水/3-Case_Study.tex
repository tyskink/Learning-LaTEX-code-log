\chapter{Case study}
\section{RISC-V}
\par Before the RISC-V, there are many RISC processor architecture like OpenRISC, SPARC and ARM. However, the OpenRISC is just an open source core with GPL protocol, which let the architecture is very difficult to be developed completely. The SPARC architecture need a huge number of registers and is hard to be used to personal computer and mobile devices. ARM focus on low-power and mobile devices help its development. But the expensive IP fees make the tradespeople start to find another alternative architecture and finally is the RISC-V Foundation in 2016.
\par The RISC-V instruction set is much simpler than ARM, due to its young and doesn�t have to be compatible to some old design. The modularity design also help the RISC-V could achieving a better adoptive ability than ARM.
\section{SpiNNaker Project}
\par There are many highlights in the SpiNNaker Project. The first one is, during this project, a spiking neural network was successfully implemented on a massively parallel processor architecture. Which means there is possible to employ other kind of neural network to MPPs. Eventually, this kind of MPP architecture could employ many kinds of neural network. The second one is the topology, which use a hexagonal architecture, which could help one neuro in the network can communication two neighbourhood in the next layer and one neighbourhood cross the next layer. 
 
Figure 3 The topology of SpiNNaker Network
